\documentclass{article}
\usepackage{amsmath}
\usepackage{amssymb}

\begin{document}

\section*{Problem Statement}

Suppose $A$ is a random variable with state space $\{0, 1\}$ and $B$ is a random variable with state space $\{1, 2, 3\}$. Suppose that the joint probability of $A$ and $B$ follows $P(A = 0, B = 1) = 1/36$, $P(A = 0, B = 2) = 5/18$, $P(A = 0, B = 3) = 1/9$, $P(A = 1, B = 1) = 1/4$ and $P(A = 1, B = 2) = 5/36$. Drawing a probability table might help.

\begin{enumerate}
    \item[(a)] [4 points] Calculate $P(A = 1, B = 3)$.
    \item[(b)] [4 points] Calculate the marginal distribution of $B$.
    \item[(c)] [8 points] Calculate the expectation and variance of $B$.
    \item[(d)] [4 points] Calculate $P(A = 0|B = 1)$.
\end{enumerate}

\section*{Solution}

Let's start by creating a joint probability table for $A$ and $B$:

\begin{table}[h]
\centering
\begin{tabular}{c|ccc|c}
$A \backslash B$ & 1 & 2 & 3 & $P(A)$ \\
\hline
0 & 1/36 & 5/18 & 1/9 & $P(A=0)$ \\
1 & 1/4 & 5/36 & $P(A=1,B=3)$ & $P(A=1)$ \\
\hline
$P(B)$ & $P(B=1)$ & $P(B=2)$ & $P(B=3)$ & 1
\end{tabular}
\end{table}

\subsection*{(a) Calculate $P(A = 1, B = 3)$}

To find $P(A = 1, B = 3)$, we can use the fact that the sum of all probabilities in the joint distribution must equal 1.

\begin{align*}
1 &= \sum_{a,b} P(A=a, B=b) \\
&= \frac{1}{36} + \frac{5}{18} + \frac{1}{9} + \frac{1}{4} + \frac{5}{36} + P(A=1, B=3) \\
P(A=1, B=3) &= 1 - (\frac{1}{36} + \frac{5}{18} + \frac{1}{9} + \frac{1}{4} + \frac{5}{36}) \\
&= 1 - (\frac{1}{36} + \frac{10}{36} + \frac{4}{36} + \frac{9}{36} + \frac{5}{36}) \\
&= 1 - \frac{29}{36} = \frac{7}{36}
\end{align*}

Therefore, $P(A = 1, B = 3) = \frac{7}{36}$.

\subsection*{(b) Calculate the marginal distribution of $B$}

The marginal distribution of $B$ can be calculated by summing over all values of $A$ for each value of $B$:

\begin{align*}
P(B = 1) &= P(A = 0, B = 1) + P(A = 1, B = 1) = \frac{1}{36} + \frac{1}{4} = \frac{10}{36} \\
P(B = 2) &= P(A = 0, B = 2) + P(A = 1, B = 2) = \frac{5}{18} + \frac{5}{36} = \frac{15}{36} \\
P(B = 3) &= P(A = 0, B = 3) + P(A = 1, B = 3) = \frac{1}{9} + \frac{7}{36} = \frac{11}{36}
\end{align*}

\subsection*{(c) Calculate the expectation and variance of $B$}

First, let's calculate the expectation of $B$:

\begin{align*}
E[B] &= \sum_{b} b \cdot P(B = b) \\
&= 1 \cdot \frac{10}{36} + 2 \cdot \frac{15}{36} + 3 \cdot \frac{11}{36} \\
&= \frac{10}{36} + \frac{30}{36} + \frac{33}{36} = \frac{73}{36} \approx 2.028
\end{align*}

Now, let's calculate the variance of $B$:

\begin{align*}
Var(B) &= E[B^2] - (E[B])^2 \\
E[B^2] &= \sum_{b} b^2 \cdot P(B = b) \\
&= 1^2 \cdot \frac{10}{36} + 2^2 \cdot \frac{15}{36} + 3^2 \cdot \frac{11}{36} \\
&= \frac{10}{36} + \frac{60}{36} + \frac{99}{36} = \frac{169}{36}
\end{align*}

Therefore,

\begin{align*}
Var(B) &= E[B^2] - (E[B])^2 \\
&= \frac{169}{36} - (\frac{73}{36})^2 \\
&= \frac{169}{36} - \frac{5329}{1296} \\
&= \frac{6084}{1296} - \frac{5329}{1296} \\
&= \frac{755}{1296} \approx 0.5825
\end{align*}

\subsection*{(d) Calculate $P(A = 0|B = 1)$}

We can use Bayes' theorem to calculate $P(A = 0|B = 1)$:

\begin{align*}
P(A = 0|B = 1) &= \frac{P(A = 0, B = 1)}{P(B = 1)} \\
&= \frac{\frac{1}{36}}{\frac{10}{36}} \\
&= \frac{1}{10} = 0.1
\end{align*}

\end{document}